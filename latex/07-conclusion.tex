\newpage
\setcounter{figure}{0}

\section{Zaključak} % (fold)
\label{sec:Zaključak}

U ovom diplomskom radu predstavljen je program za izgradnju mreže
trokuta mesh-reconstruction. Program se upotrebljava nad oblakom točaka
prikupljenim RGBDSlam programom. Ispitana je funkcionalnosti i
kvaliteta postupka snimanjem i izgradnjom nekoliko 3D modela scene
pomoću 3D kamere. Tijekom izrade rada postajalo je niz prepreka koje je
trebalo riješiti prije rada na konkretnom zadataku. Glavne prepreke su
bile prevođenje, instalacija, proučavanje i pokretanja programa RGBDSlam
te postavljanje radne okoline za pisanje programa. 

Predstavljeni postupak snimanja i izgradnje 3D modela ima svoje
prednosti i nedostatke. Nedostatci su vezani uz ograničenja korištenih
tehnologija i algoritama. Npr. snimanje Kinectom ograničeno je na
prostore u unutrašnjosti, uspješnost snimanja RGBDSlam programom uvelike
ovisi o detekciji i sparivanju značajki, Poisson algoritam ne uzima u
obzir informaciju o boji itd. Prednosti je svakako nabavna cijena
kamere koja se može nabaviti za manje od tisuću kuna. Isto tako svi
korišteni programi objavljeni su pod slobodnim licencama te su dostupni
za proučavanje, poboljšavanje i upotrebu u bilo koje svrhe.  

Postoji potencijal opisnog postupka izgradnje odnosno generalna ideja
ima budućnosti i načine upotrebe. Sve veća dostupnost 3D printera za
generalnu populaciju stvara potražnju za jednostavnim i jeftinim
rješenjem skeniranja 3D scena i objekata. Osim toga dizajneri FPS (engl.
\textit{First Person Shooter}) igara mogli bi koristiti sličan postupak
za skeniranje prostorija od kojih bi izgrađivali 3D modele.

Mogućnosti za poboljšanja postoje na svim razinama. Najveći utjecaj na
krajnje rezultate bilo bi proširivanje Poisson algoritma logikom za
očuvanje informacije o boji. Isto tako algoritamu bi se mogla dodati
logika koja uklanja zatvaranje velikih rupa, ideja je da se prođe još
jednom kroz izrađenu mrežu te pronađu i odbace veliki trokuti. Kod
snimanja scene jednostavno poboljšanje bilo bi korištenje jačeg računala
i automatskog snimanja.

% section Zaključak (end)
