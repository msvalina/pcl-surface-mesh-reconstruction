\newpage

\setcounter{page}{1}
\setcounter{figure}{0}
\section{Uvod}% (fold)
\label{sec:Uvod}

Pojava i dostupnost jeftinog i kvalitetnog 3D senzora Microsoft Kinect
kamere 2010. godine za igračku konzolu Xbox 360 uvelike je doprinjela
razvoju računalnog vida. Senzor u VGA rezoluciji frekvencijom od 30Hz
daje sliku u boji te informaciju o dubini točaka slike, tj. njihovoj
udaljenosti od kamere. Time omogućava da se na prirodniji način pristupi
rješavanju osnovnih problema računalnog vida.  Upravo to se i dogodilo
te su znanstvenici, programeri i hakeri razvili upravljačke programa,
alate i algoritme za korištenje Kinect senzora i sličnih uređaja za
prikupljanje oblaka točaka. Većina izrađenog softvera je objavljena pod
slobodnim licencama koje omogućavaju slobodu upotrebe programa u bilo
koje svrhe, slobodu proučavanja i primjenjivanja stečenog znanja,
slobodu distribuiranja kopija u cijelosti ili u dijelovima te slobodu
mijenjanja, poboljšavanja i distribuiranja derivacijskih programa.
Upravo ti programi su postavili temlje za ovaj diplomski rad.

Osim Kinect kamere u radu se koristi jos niz suvremenih tehnologija i
algoritama: ROS (Robot Operating System) biblioteka i alati, biblioteka
Pointcloud, istovremena lokalizacija i mapiranje te Poisson algoritam za
rekonstrukciju površine.

Rad se sastoji iz tri dijela. Prvi dio odnosi se na pregled korištenih
tehnologija i algoritama. Drugi dio je praktični dio i govori o
izgradnji 3D modela scene. Prvo je objašnjen postupak snimanja scene 3D
kamerom i RGBDSlam programom, a zatim je dan pregled izgradnje 3D modela
scene mrežom trokuta. Treći dio rada prikazuje rezultate snimanja i
izrade te ispituje funkcionalnost i kvalitetu postupka.

\newpage
\subsection{Zadatak diplomskog rada} % (fold)
\label{sub:Zadatak diplomskog rada}

Program RGBDSLAM raspoloživ u okviru incijative dijeljenja algoritama
OpenSLAM omogućava izgradnju 3D modela objekata i scena pomoću 3D
kamere. Zadatak ovog rada je razviti program za izgradnju 3D modela u
obliku mreže trokuta koristeći programsku biblioteku PointCloud Library.
Kombinacijom ova dva programa mogu se izgraditi 3D modeli objekata i
scena snimljenih iz više pogleda. Zadatak je ispitati funkcionalnost
navedenog postupka kao i kvalitetu dobivenog rezultata izgradnjom
nekoliko 3D modela objekata i scena. Na
slici~\ref{fig:project-description} grafički je prikazan zadatak
diplomskog rada.

\begin{figure}[h]
\centering
\includegraphics[scale=0.35]{figures/project-description.jpeg}
\caption[]{Grafički prikaz projekta upotrebom Standfordovog
zeca\footnotemark[1]}
\label{fig:project-description}
\end{figure}

\footnotetext[1]{%
Standford Bunny 3D model su originalno konstruirali 1994 Greg Turk i
Marc Levoy i od tada je postao najčešće upotrebljevani model za
testiranje tehnika u računalnoj grafici. \url{http://www.gvu.gatech.%
edu/people/faculty/greg.turk/bunny/bunny.html}
}

% subsection Zadatak diplomskog rada (end)
% section Uvod (end)
