\newpage

\section{Izgradnja 3D modela scene} % (fold)
\label{sec:Izgradnja 3D modela scene}

\subsection{Snimanje scene 3D kamerom i RGBDSlam programom} % (fold)
\label{sub:Snimanje scene 3D kamerom i RGBDSlam programom}

% subsection Snimanje scene 3D kamerom i RGBDSlam programom (end)

\subsection{Izgradnja 3D modela scene pomoću mreže trokuta} % (fold)
\label{sub:Izgradnja 3D modela scene pomoću mreže trokuta}

Izgradnja 3D modela scene pomoću mreže trokuta je implementirana u
programu nazvanom \texttt{mesh-reconstruction}.\footnote{
Program \texttt{mesh-reconstruction} je slobodan program dostupan pod
uvijetima MIT licence. Izvorni kod se nalazi na DVD-u te na web stranici
http://github.com/msvalina/}      
Program se intenzivno oslanja na biblioteku PointCloud koja je opisana u
potpoglavlju \ref{sub:Biblioteka Pointcloud} Kao što je vidljivo iz
grafikona \ref{fig:flowchart} program je podijeljen u pet osnovnih
funkcija:
\begin{itemize}
    \item Učitavanje oblaka točaka snimljenih RGBDSlam programom.
    \item Reduciranje oblaka točaka.
    \item Uklanjanje pogrešaka pri mjerenju.
    \item Izrađivanje i zapisivanje mreže trokuta.
    \item Prikaz mreže trokuta.
\end{itemize}

\begin{figure}[h]
\renewcommand{\figurename}{Grafikon}
\centering
\includegraphics[scale=0.5]{figures/flowchart.pdf}
\caption{Dijagram toka programa \texttt{mesh-reconstruction} }
\label{fig:flowchart}
\end{figure}

U sljedećim potpoglavljima dan je pregled funkcija i PCL klasa nad kojim
se baziraju.

\subsubsection{Učitavanje oblaka točaka} % (fold)
\label{ssub:Učitavanje oblaka točaka}
Program učitava podatke na početku svake funkcije, te ih zapisuje na
izlazu iz funkcije kako bih prije i poslije svake operacije bio dostupan
oblak točaka. Za to koristi \texttt{PCDReader} i \texttt{PCDWriter} klase.
Primjer takvog koda se nalazi u ispisu koda~\ref{lstUcitavanjeOblaka}

\begin{lstlisting}[label=lstUcitavanjeOblaka, caption={Primjer izvornog
koda za učitavanje oblaka točaka}]
    // Init cloud variables 
    pcl::PCLPointCloud2::Ptr cloud (new pcl::PCLPointCloud2());
    pcl::PCLPointCloud2::Ptr cloud_filtered (new pcl::PCLPointCloud2());

    // Fill in the cloud data
    pcl::PCDReader reader;
    reader.read ("pointcloud.pcd", *cloud);
    /* 
     * Do something with cloud
     */
    // Write cloud to a file
    pcl::PCDWriter writer;
    writer.write ("pointcloud-downsampled.pcd",
            *cloud_filtered, Eigen::Vector4f::Zero(),
            Eigen::Quaternionf::Identity(), false);
\end{lstlisting}

% subsubsection Učitavanje oblaka točaka (end)

\subsubsection{Reduciranje oblaka točaka} % (fold)
\label{ssub:Reduciranje oblaka točaka}
Reduciranje obalaka ne unosi gubitak informacija, a izvodi se zbog lakše
daljnje obrade oblaka. Izvodi se pomoću \texttt{VoxelGrid} klase i
implementirano je u \texttt{downsample()} funkciji. Dijelovi funkcije
prikazani su u ispisu koda~\ref{lstReduciranje}
\texttt{VoxelGrid} dolazi od riječi \textit{volume pixel grid} i
predstavlja niz malih kocaka u prostoru.

\begin{lstlisting}[label=lstReduciranje, caption={Dio izvornog koda za
reduciranje točaka iz funkcije \texttt{downsample()} }]
    // Create the filtering object
    pcl::VoxelGrid<pcl::PCLPointCloud2> vg;
    vg.setInputCloud (cloud);
    // voxel size to be 1cm^3
    vg.setLeafSize (0.01f, 0.01f, 0.01f);
    vg.filter (*cloud_filtered);
\end{lstlisting}

Kao što se vidi iz ispisa koda~\ref{lstReduciranje} nakon kreiranja
objekta \texttt{vg} predaje mu se oblak točaka nad kojim se vrši
reduciranje. Postavlja se veličina kocke (\textit{voxel}) u našem
slučaju to je 3cm\textsuperscript{3}. Nad tim oblakom prilikom
filtriranja će se kreirati mreža kocaka te će se sve točke unutar jedne
kocke zamjeniti centralnom točkom. Tim postupkom se značajno smanjuje
broj točaka u oblaku.

% subsubsection Reduciranje oblaka točaka (end)

\subsubsection{Uklanjanje pogrešaka pri mjerenju} % (fold)
\label{ssub:Uklanjanje pogrešaka pri mjerenju}
Šum pri mjerenju je sastavni dio svakog mjernog uređaja pa tako i
Kineckt kamere. PointCloud biblioteka ima ugrađenu
\texttt{StatisticalOutlierRemoval} klasu koja uklanja šum te je
implementirana u funkciji \texttt{remove\_outlieres()}.
Iz ispisa koda~\ref{lstUklanjanje} se vidi kako se klasa koristi.

% minipage ensures that listing won't be split between pages
\begin{minipage}{\textwidth}
\begin{lstlisting}[label=lstUklanjanje, caption={Dio izvornog koda za
uklanjanje pogrešaka pri mjerenju iz funkcije \texttt{remove\_outliers()} }]
    // Create the filtering object
    pcl::StatisticalOutlierRemoval<pcl::PCLPointCloud2> sor;
    sor.setInputCloud (cloud);
    // Set number of neighbors to analyze
    sor.setMeanK (50);
    sor.setStddevMulThresh (1.0);
    sor.filter (*cloud_filtered);
\end{lstlisting}
\end{minipage}

Nakon kreiranja objekta \texttt{sor} i predavanja oblaka postavljena su
još dva parametra. Prvi \texttt{setMeanK} je broj susjednih točaka koje
će filter analizirat. Drugi \texttt{setStddevMulThresh} pak kaže da će
sve točke u okolini ispitane točke čije su udaljenosti veće od jedne
standardne devijacije očekivane udaljenosti biti označne kao šum
(\textit{outlier}) i odbačene.

% subsubsection Uklanjanje pogrešaka pri mjerenju (end)

\subsubsection{Izrađivanje i zapisivanje mreže trokuta} % (fold)
\label{ssub:Izradivanje i zapisivanje mreže trokuta}

Nakon pripreme oblaka točaka funkcijama \texttt{downsample()} i
\texttt{remove\_outliers()} slijedi izrađivanje mreže trokuta unutar
funkcije \texttt{mesh\_reconstruction()}. 

% subsubsection Izrađivanje i zapisivanje mreže trokuta (end)

% subsection Izgradnja 3D modela scene pomoću mreže trokuta (end)

% section Izgradnja 3D modela scene (end)
