\newpage
\setcounter{figure}{0}

\section{Pregled korištenih tehnologija i algoritama} % (fold)
\label{sec:Tehnologija i teorija}


\subsection{Microsof Kineckt 3D kamera} % (fold)
\label{sub:Microsof Kineckt 3D kamera}

% subsection Microsof Kineckt 3D kamera (end)

\subsection{ROS biblioteka i alati} % (fold)
\label{sub:ROS biblioteka i alati}

% subsection ROS biblioteka i alati (end)

\subsection{Biblioteka Pointcloud} % (fold)
\label{sub:Biblioteka Pointcloud}

% subsection Biblioteka Pointcloud (end)


\subsection{Istovremena lokalizacija i mapiranje} % (fold)
\label{sub:Slam}

% subsection Slam (end)

\newpage
\subsection{Poisson algoritam za rekonstrukciju površine} % (fold)
\label{sub:Poisson}
Poisson algoritam za rekonstrukciju površine~\cite{Kazhdan:2006}
razvijen je suradnjom Michaela Kazhdana i Matthewa Bolitha s Johns
Hopkins sveučilišta u Baltimoru i Huguesa Hoppea iz Microsoft Researcha
u Redmondu. Također Kazhdan i Bolitho su implementirali\footnotemark[1]
Poisson algoritam i objavili kod pod BSD licencom. Na osnovu tog rada
algoritam je dodan i u PCL biblioteku.

U ovom potpoglavlju nalazi se osnovna ideja i kratak matematički pregled
algoritma. Opisano je ograničenje algoritma te parametri kojima se može
upravljati rekonstrukcija površine.

\footnotetext[1]{%
Originalna implementacija Poisson algoritma se nalazi na 
\url{http://www.cs.jhu.edu/~misha/Code/PoissonRecon/Version5.5/}}

% Rekonstrukcija 3D površina iz uzorka točaka je dobro proučavan problem u
% računalnoj grafici. Ona omogućava namještanje skeniranih
% podataka, ispunjavanje površinskih rupa i ponovnu izgradnju postojećih
% modela.

\newpage
\subsubsection{Osnovna ideja i kratak matematički pregled} % (fold)
\label{ssub:Osnovna ideja i kratak matematički pregled}

Poisson algoritam pristupa problemu rekonstrukcije površine rješavanjem
Poissonove jednadžbe. To čini upotrebom metode implicitne funkcije.
Točnije računanjem 3D indikacijske funkcije \(\chi\) (definiranom s 1 u
točkama unutar modela, odnosno s 0 u točkama izvan) i dohvaćanjem
rekonstruktruirane površine izvlačenjem odgovarajuće iso-površine.

Algoritam se oslanja na ideju da postoji cjelovita veza između
orijentiranih normala uzetih s površine modela i indikacijske funkcije
modela. Točnije, gradijent indikacijske funkcije je polje vektora koje
je uglavnom popunjeno nulama (jer je indikacijska funkcija uglavnom
konstantna), osim kod točaka blizu površine gdje je jednako unutrašnjim
normalama površine. Stoga, uzorci orijentiranih normala mogu biti
promatrani kao gradijent modelove indikacijske funkcije kao što je
prikazano na slici~\ref{fig:poisson-reconstruction.png}

\begin{figure}[h]
\centering
\includegraphics[scale=0.35]{figures/poisson-reconstruction.png}
\caption[]{Prikaz Poisson rekonstrukcije u 2D,
    izvor:~\cite{Kazhdan:2006}}
\label{fig:poisson-reconstruction.png}
\end{figure}

Problem računanja indikacijske funkcije se svodi na invertiranje
operatora gradijenta, odnosno pronalazak funkcije skalara \(\chi\) čiji
gradijent najbolje aproksimira polje vektora \(\vec{V}\) definirano
uzorcima, odnosno 

\begin{equation*}
min_\chi \|\nabla\chi - \vec{V}\|.
\end{equation*}

Ako se primjeni operator divergencije, tada se taj problem pretvara u
standardni Poissonov problem: računanje funkcije skalara \(\chi\) čiji
laplasijan (divergencija gradijenta) je jednak divergenciji polja
vektora \(\vec{V}\),

\begin{equation*}
\Delta \chi \equiv \nabla \cdot \nabla\chi = \nabla \cdot \vec{V}.
\end{equation*}

Predstavljanje rekonstrukciju površine kao Poissonov problem pruža
nekoliko prednosti. Mnoge implicitne metode namještanja površina
segementiraju podatke u regije za lokalno namještanje i onda te lokalne
aproksimacije spajaju upotrebom funkcija stapanja. Za razliku od njih,
Poisson rekonstrukcija je globalno rješenje koje razmatra sve podatke
odjednom, bez upotrebe heurstičkih podijela i stapanja. Zbog toga
Poisson rekonstrukcija kreira izrazito glatku površinu koja robusno
aproksimira šumovite podatke.  

Za izvlačenje iso-površine Poisson algoritam koristi Marching Cubes
algoritam~\cite{Lorensen87marchingcubes} koji kreira octree strukturu
podataka za prikaz površine.  Kao što se vidi na
slici~\ref{fig:poisson-marching-cubes.png} Marching Cubes algoritam
dijeli oblak točaka u mrežu voxela marširajući kroz oblak i analizira
koje točke čine iso-površinu objekta.  Detektiranjem koji rubovi voxela
presjecaju iso-površinu modela algoritam kreira mrežu trokuta. Više
informacija o izvlačenju površine se mogu pronaći u radu “Unconstrained
Isosurface Extraction on Arbitrary Octrees” Michaela
Kahzdana~\cite{Kazhdan:2007}

\begin{figure}[h]
\centering
\includegraphics[scale=0.8]{figures/poisson-marching-cubes.png}
\caption[]{Prikaz Marching cubes algoritma, izvor:~\cite{Kazhdan:2007}}
\label{fig:poisson-marching-cubes.png}
\end{figure}

% subsubsection Osnovna ideja i kratak matematički pregled (end)

\subsubsection{Ograničenje Poisson algoritma} % (fold)
\label{ssub:Ograničenje Poisson algoritma}

Ograničenje implementacije Poisson algoritma je u tome što ne uzima u
obzir informacije asocirane s načinom stjecanja oblaka točaka.
Slika~\ref{fig:poisson-buddha.png} pokazuje kip Bude i vidi se primjer takvog
ograničenja. Budući da nema točaka između Budinih nogu, Poisson
algoritam spaja te dvije regije. Algoritam se može unaprijediti
ugradnjom dodatne informacije poput vidokruga i na taj način izbjeći to
ograničenje.

\begin{figure}[h]
\centering
\includegraphics[scale=0.20]{figures/poisson-buddha.png}
\caption[]{Rekonstrukcija modela ``Happy Buddha'' 
VRIP\footnotemark[2] algoritam (lijevo) i Poisson algoritma (desno),
izvor:~\cite{Kazhdan:2006}}
\label{fig:poisson-buddha.png}
\end{figure}

\footnotetext[2]{%
VRIP - Volumetric Range Image Processing~\cite{Curless:1996VRIP}}

% subsubsection Ograničenje Poisson algoritma (end)

\newpage
\subsubsection{Parametri Poisson algoritma} % (fold)
\label{ssub:Parametri Poisson algoritma}
Postoji nekoliko parametara koji utječu na rezultat rekonstrukcije.
\begin{itemize}
    \item \texttt{Depth:} dubina octree stabla koje se koristi za
        rekonstrukciju. Zadana vrijednost 8.
    \item \texttt{SolverDivide:} postavlja dubinu kod kojeg bloka
        Gauss-Seidel metoda riješava Laplasovu jednadžbu. Zadana
        vrijednost 8.
    \item \texttt{IsoDivide:} postavlja dubinu kod kojeg bloka
        ekstraktor iso-površine izvlači iso-površinu. Zadana vrijednost
        8.
    \item \texttt{SamplesPerNode:} postavlja minimalni broj točaka koje
        se trebaju nalaziti unutar octree čvora kako se octree
        konstrukcija prilagođava gustoći sempliranja. Za podatke bez
        šuma 1 - 5, sa šumom 15 - 20.
    \item \texttt{Scale:} omjer između promjera kocke korištne za
        rekonstrukciju i promjera kocke koja omeđuje uzorke. Zadana
        vrijednost 1.25.
    \item \texttt{Confidence:} postavljanje zastavice govori
        rekonstruktorciji da koristi veličinu normala kao informaciju o
        pouzdanosti. Ako nije postavljena sve normale se normaliziraju
        prije rekonstrukcije.
\end{itemize}

Od nabrojanih parametara najvažniji utjecaj na generiranu mrežu imaju
\texttt{SamplesPerNode} i \texttt{Depth}. Veća dubina octree stabla
rezultira većom precinosti mreže voxela jer Marching Cubes algoritam
ulazi dublje u stablo. Manja dubina (između 5 i 7) daje glađi model ali
s manje detalja. \texttt{SamplesPerNode} parametar definira koliko će
točaka Marchin Cubes algoritma staviti u jedan čvor rezultantnog octree
stabla. Ako algoritam radi s podacima punim šuma velik uzorak točaka (15
- 20) po čvoru pruža glađenje ali se gube detalji. Dok rad s malim
vrijednostima (1 - 5) održava razinu detalja visokom. Velike vrijednosti
reduciraju kranji broj vrhova poligona, dok male održavaju broj vrhova
visokim.

% subsubsection Parametri Poisson algoritma (end)

% subsection Poisson (end)

% section Tehnologija i teorija (end)
